\phantomsection\section*{ВВЕДЕНИЕ}\addcontentsline{toc}{section}{ВВЕДЕНИЕ}

Анализ структуры документов (Document Layout Analysis, DLA) играет ключевую роль в обработке научно-технических текстов.
Такие документы обладают четкой структурой, включающей заголовки, авторов, аннотации, разделы, формулы, таблицы, графики и рисунки~\cite{visual-2019, dla-survey, dsaa-survey, dia-survey}.
Выявление этих элементов и их логических связей позволяет не только упрощать индексирование и поиск информации, но и улучшать автоматическую обработку текстов, включая аннотирование, реферирование и анализ содержимого.

Документ можно представить в виде иерархии физических модулей (страницы, колонки, абзацы, строки, слова, изображения) или логических модулей (заголовки, авторы, аффилиации, аннотации, разделы, библиография)~\cite{dla-book}.

Эффективный анализ структуры документов обеспечивает удобную навигацию по тексту, облегчает его разметку и позволяет быстро извлекать необходимые сведения~\cite{dla-book}.

Целью данной работы является классификация методов выделения составных частей научного текста.

Для достижения поставленной цели необходимо решить следующие задачи:
\begin{itemize}
    \item провести анализ предметных областей анализа структуры документов и научно-технических текстов;
    \item провести обзор существующих методов выделения составных частей научного текста;
    \item сформулировать критерии сравнения описанных методов;
    \item провести классификацию описанных методов по сформулированным критериям.
\end{itemize}
